\documentclass[9pt]{beamer} 
\usetheme{CambridgeUS}
\usepackage[utf8]{inputenc}
\usefonttheme{professionalfonts}
\usecolortheme{dolphin}
\setbeamertemplate{items}[triangle]
\setbeamercolor{alerted text}{fg=orange}
\usepackage{times}
\usepackage{amsmath}
\usepackage{verbatim}
\usepackage{cancel}
\renewcommand{\CancelColor}{\color{orange}}

\newcommand{\redbox}[1]{\colorbox{red!50}{$\displaystyle#1$}}
\newcommand{\greenbox}[1]{\colorbox{green!50}{$\displaystyle#1$}}
\newcommand{\orangebox}[1]{\colorbox{orange!50}{$\displaystyle#1$}}

\author[C.~Evoli]{Carmelo Evoli}
\institute[GSSI]{Gran Sasso Science Institute @L'Aquila}
\title[CR in self-generated turbulence]{New consequences of cosmic-ray propagation in self-generated turbulence}
\date[CIII Congresso SIF 2017]{CIII Congresso Nazionale della SIF @Trento}
\titlegraphic{\includegraphics[width=1cm]{logo_gssi}}

\begin{document}

%%%%%%%%%%%%%%%%%%%%%%%%%%%%%%%%%%%%%%%%%%%%%%%%%%%%%%%%%%%%%%%%%%%%%%%%
\begin{frame}
\titlepage
\end{frame}

%%%%%%%%%%%%%%%%%%%%%%%%%%%%%%%%%%%%%%%%%%%%%%%%%%%%%%%%%%%%%%%%%%%%%%%%
\begin{frame}{Outline}
\tableofcontents
\end{frame}

%%%%%%%%%%%%%%%%%%%%%%%%%%%%%%%%%%%%%%%%%%%%%%%%%%%%%%%%%%%%%%%%%%%%%%%%
\begin{frame}{The diffusion-advection equation}

We want to solve the diffusion-advection equation for a cosmic-ray species whose density per unit of energy is $N$:
%
\begin{equation*}
\frac{\partial N}{\partial t} 
+\frac{\partial}{\partial z} \left[ D_{zz}(z) \frac{\partial N}{\partial z} \right]
+v_z \frac{\partial N}{\partial z}
= Q_{\rm SNR}(z)
\end{equation*}
%
having boundary conditions $N(z \pm H) = 0$.

We assume an {\color{orange}inhomogeneous} diffusion coefficient, hence we cannot solve it analytically:
%
\begin{eqnarray*}
D_{zz} & = & D_0 \left( \frac{z}{z_d} \right)^{2} \\
Q_{\rm SNR} & = & Q_0 \delta(z) \left( \frac{p}{\rm GeV/c} \right)^{-\alpha} \\
v_z & = & v_0
\end{eqnarray*}
\end{frame}

%%%%%%%%%%%%%%%%%%%%%%%%%%%%%%%%%%%%%%%%%%%%%%%%%%%%%%%%%%%%%%%%%%%%%%%%
\begin{frame}{title}

We introduce the space and time grids: 
%
\begin{eqnarray*}
t_n & = & n \Delta t \, , \,\,\,\,  n = 0, 1, \dots  \\ 
z_j & = & i \Delta z \, , \,\,\,\,  i = 0, 1, \dots, N-1
\end{eqnarray*}
%
where
%
$t_0 = 0$, while $z_0 = -H$, $z_{N-1} = +H$, therefore:
%
\begin{equation*}
\Delta z = \frac{2H}{N - 1}
\end{equation*}

CR density at a given position and time can be written on this lattice as
%
\begin{equation*}
N(z, t) \rightarrow N_i^n
\end{equation*}

\end{frame}

%%%%%%%%%%%%%%%%%%%%%%%%%%%%%%%%%%%%%%%%%%%%%%%%%%%%%%%%%%%%%%%%%%%%%%%%
\begin{frame}{Derivative approximations}

We can approximate the time derivative as:
%
\begin{eqnarray*}
\frac{\partial N}{\partial t} & \sim & \frac{N_i^{n+1} - N_i^n}{\Delta t} + {\mathcal O}(\Delta t^p) \\
\end{eqnarray*}
%
while, for the space 1st-order derivative we have different choices:
%
\begin{eqnarray*}
\frac{\partial N}{\partial z} & \sim & \frac{N^n_{i+1} - N^n_i}{\Delta t} + {\mathcal O}(\Delta z^p) \qquad \text{forward}\\
%\frac{\partial N}{\partial z} & \sim & \frac{N^n_{i+1} - N^n_{i-1}}{2\Delta t} + {\mathcal O}(\Delta z^p) \qquad \text{central} \\
\frac{\partial N}{\partial z} & \sim & \frac{N^n_i - N^{n}_{i-1}}{\Delta t} + {\mathcal O}(\Delta z^p) \qquad \text{backward} \\
\end{eqnarray*}
%
and for the second order derivative:
%
\begin{eqnarray*}
\frac{\partial^2 N}{\partial z^2} & \sim & \frac{N^n_{i+1} - 2N^n_i + N^{n}_{i-1}}{2 \Delta z^2} + {\mathcal O}(\Delta z^p) \qquad \text{central} 
\end{eqnarray*}

{\color{orange}Exercises:}
%
\begin{itemize}
\item find $p$ in the different schemes \\
\item find the central scheme for the 1st-order derivative \\
\item find the forward scheme for the 2nd-order derivative \\
\end{itemize}

\end{frame}

%%%%%%%%%%%%%%%%%%%%%%%%%%%%%%%%%%%%%%%%%%%%%%%%%%%%%%%%%%%%%%%%%%%%%%%%
\begin{frame}{title}

Schematically the transport equation can be now written as:
%
\begin{equation*}
\frac{\partial N}{\partial t} = {\mathcal L}N + Q
\end{equation*}
%
where ${\mathcal L}$ is the operator which defines the transport equation.

In its discretized version, the transport equation becomes:
%
\begin{equation*} 
\frac{N_i^{n+1} - N_i^n}{\Delta t} = {\mathcal L}_i + Q_i \underset{\rm o.s.}{=}
 \sum_k {\mathcal L}_i^k + \frac{Q_i}{m} 
\end{equation*}

By comparison with Eq.~ we have:
%
\begin{eqnarray*}
{\mathcal L}_d & = & \frac{\partial}{\partial z} \left[ D_{zz}(z) \frac{\partial N}{\partial z} \right]\\
{\mathcal L}_a & = & v_z \frac{\partial N}{\partial z} \\
\end{eqnarray*}

\end{frame}

%%%%%%%%%%%%%%%%%%%%%%%%%%%%%%%%%%%%%%%%%%%%%%%%%%%%%%%%%%%%%%%%%%%%%%%%
\begin{frame}{Energy losses}

We want now the steady-state solution of the following equation (valid for $E \gg m$):
%
\begin{equation*}
\frac{\partial N}{\partial t} + \frac{\partial}{\partial E} \left[ b N \right] = Q(E) \qquad b \equiv \frac{dE}{dt} < 0
\end{equation*}
%
with b.c.~$N(E = E_{\rm max}) = 0$.

The analytical solution is obtained by assuming power-laws for both the loss term rate and for the source term:
%
\begin{equation*}
b(E) = -b_0 \left( \frac{E}{\rm GeV} \right)^2 \qquad b_0 > 0
\end{equation*}
%
\begin{equation*}
Q(p) = Q_0 \left( \frac{E}{\rm GeV} \right)^{-\alpha} \qquad \alpha > 2
\end{equation*}
%
therefore
%
\begin{equation*}
N_a(E) = \frac{Q_0}{\alpha - 1} \frac{\rm GeV}{b_0} \left[ \left( \frac{E}{\rm GeV} \right)^{-\alpha-1} - \left( \frac{E_{\rm max}}{\rm GeV} \right)^{-\alpha + 1} \left( \frac{E}{\rm GeV} \right)^{-2} \right]
\end{equation*}

\end{frame}

%%%%%%%%%%%%%%%%%%%%%%%%%%%%%%%%%%%%%%%%%%%%%%%%%%%%%%%%%%%%%%%%%%%%%%%%
\begin{frame}{Energy discretisation}

The energy grid is chosen such that $E_{i+1} = (1 + \epsilon) E_i \, \forall i$ with $\epsilon > 0$:
%
\begin{equation*}
E_i = (1 + \epsilon)^i E_0 \qquad i = 0, \dots, N_E - 1
\end{equation*}
%
Therefore:
%
\begin{equation*}
N(E,t) \rightarrow N_i^n
\end{equation*}

This choice makes integrals over energy very easy:
%
\begin{equation}\label{integral}
E_T = \int_{E_0}^{E_{\rm max}} dE \, E N(E,t) \sim \log(1+\epsilon) \sum_k E_k^2 N_k^n
\end{equation}


{\color{orange}Exercises:}
%
\begin{itemize}
\item find $\epsilon$ given $E_0$, $E_{\rm max}$, $N_E$ \\
\item assume $N(E,t) \propto E^{-2.7}$ and estimate the accuracy of the numerical integral in~(\ref{integral}) as a function of $\epsilon$ \\
\end{itemize}
\end{frame}

%%%%%%%%%%%%%%%%%%%%%%%%%%%%%%%%%%%%%%%%%%%%%%%%%%%%%%%%%%%%%%%%%%%%%%%%
\begin{frame}{Upwind scheme}

It corresponds to an {\color{orange}advection} (with $v < 0$) along the energy axis, therefore we use 
%
\begin{equation*}
\frac{N^{n+1}_i - N^n_i}{\Delta t} = \frac{b_{i+1} N_{i+1}^n - b_i N_i^n}{E_{i+1} - E_i} + Q_i
\end{equation*}
%
\begin{equation*}
N^{n+1}_i = N^n_i + \Delta t \frac{b_{i+1}}{E_{i+1} - E_i} N_{i+1}^n - \frac{b_i}{E_{i+1} - E_i} N_i^n} + Q_i \Delta t 
\end{equation*}



\end{frame}


\end{document}